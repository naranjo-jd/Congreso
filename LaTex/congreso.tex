\documentclass{article}
\usepackage{graphicx}
\usepackage{amsmath, amssymb}
\usepackage[utf8]{inputenc}

\title{Congreso}
\author{ }
\date{}

\begin{document}

\maketitle

\section{Grafos}

Un grafo es una tupla $G = (V, E)$, donde $V$ es un conjunto no vacío de elementos llamados vértices (o nodos) y $E$ es un conjunto de pares de vértices, llamados aristas.\\

Un grafo $G = (V, E)$ es dirigido si $E \subseteq V \times V$

Un grafo ponderado es una tripleta $G = (V, E, \omega)$, donde $\omega$ es una función $\omega: E \rightarrow \mathbb{R}$.

Sea $G$ un grafo, donde $G(V) = \lbrace v_1, ..., v_n \rbrace$. La matriz de adyacencia de $G$ es una matriz cuadrada $A = [a_{ij}]$ de $n \times n$ definida por

\[
a_{ij} =
\begin{cases} 
\ 1, & \text{si } \lbrace v_i , v_j\rbrace \in E \\
\ 0, & \text{en caso contrario}
\end{cases}
\]

Si $G$ es un grafo ponderado, su matriz de adyacencia ponderada está dada por

\[
a_{ij} =
\begin{cases} 
\ \omega( v_i , v_j), & \text{si } (v_i , v_j) \in E \\
\ 0, & \text{en caso contrario}
\end{cases}
\]

\section{Energía}

Sea $G = (V, E)$ un grafo con matriz de adyacencia $A(G)$. Decimos que el polinomio característico de $G$ es el polinomio caracteristico de su matriz de adyacencia $A(G)$ dado por $\text{det}(xI_n-A(G))$, donde $I_n$ es la matriz identidad de orden $n$. Denotamos al polinomio caracteristico de $G$ con $\phi(G, x)$. Los autovalores del grafo $G$ son los autovalores de su matriz de adyacencia $A(G)$. Dado que $A(G)$ es una matriz simetrica, sus autovalores son reales. Llamamos espectro de $G$ al conjunto de autovalores del grafo y lo denotamos por $\text{Spec}(G)$. 

$$\text{Spec}(G) = \lbrace \lambda \in \mathbb{R} : \phi(G, \lambda) = 0 \rbrace$$

Sea $G$ un grafo con $n$ vertices. Definimos la energia de $G$ como

$$\mathcal{E} = \mathcal{E}(G) = \sum_{\lambda \in \text{Spec(G)}} |\lambda|$$

El grado de un vértice es el número de vértices conectados a él y lo denotamos $\text{deg}(v)$. Dado un grafo $G$ con $V(G) = \lbrace v_1, ... , v_n \rbrace$ definimos $\Delta(G)$ como la matriz diagonal con entradas

$$\delta_{ii} = \text{deg}(v_i)$$

La Matriz Laplaciana de $G$, denotada $L(G)$, está definida como

$$L(G) = \Delta(G) - A(G)$$

Sea $G$ un grafo con $|V(G)| = n$, $|E(G)| = m$ y $S = \text{Spec}(L(G) - \frac{2m}{n}I_n)$. La energía Laplaciana de $G$, denotada $\mathcal{E}_L$, está dada por

$$\mathcal{E}_L = \sum_{\lambda \in S} |\lambda|$$

Una \textbf{red neuronal} es un modelo computacional inspirado en el cerebro humano, utilizado en aprendizaje automático e inteligencia artificial para reconocer patrones y tomar decisiones. Consiste en capas de unidades interconectadas llamadas \textbf{neuronas}, que procesan datos de entrada y transmiten información a través de conexiones ponderadas.

\section*{Relación con la Teoría de Grafos}

Una red neuronal puede verse como un \textbf{grafo dirigido}, donde:
\begin{itemize}
    \item \textbf{Nodos (vértices)} representan las neuronas.
    \item \textbf{Aristas dirigidas} representan las conexiones ponderadas entre neuronas.
\end{itemize}

La estructura de una red neuronal se relaciona con la teoría de grafos en varios aspectos:
\begin{enumerate}
    \item \textbf{Redes Feedforward}: Forman un \textbf{grafo dirigido acíclico (DAG)} donde la información fluye en una sola dirección, desde la entrada hasta la salida, sin ciclos.
    \item \textbf{Redes Neuronales Recurrentes (RNNs)}: Introducen ciclos, haciendo que el grafo sea \textbf{cíclico}, lo que permite la memoria y dependencias temporales en los datos.
    \item \textbf{Redes Neuronales de Grafos (GNNs)}: Utilizan estructuras de \textbf{grafos} como entrada, aprendiendo patrones sobre los nodos y aristas, lo que es útil para redes sociales, estructuras moleculares, entre otros.
\end{enumerate}

\end{document}
